% intro

\section{Introduction}
\label{sec:intro}
% What we do : SDD + Terms

Data Decision Diagrams (DDD) \cite{Couvreur02DDD}
 are a directed acyclic graph structure that manipulates(\textit{a priori} unbounded)
integer domain variables, and which offers a flexible and
compositional definition of operations through inductive
homomorphisms. 

 The DDD definition has been extended in \cite{forte05} to
introduce hierarchy in the data structure. We defined Set Decision
Diagrams, in which a variable's domain is a \emph{set of
values}. Concretely, it means the arcs of an SDD may be labeled with
an SDD (or a DDD), introducing the possibility of arbitrary depth
nesting in the data structure.  

%%% Le probleme  de Didier

\fixme{introduire le probl�me de r�ecriture, plus limites de l'approche traditionnelle Prolog}

% What problems solved : cache, identical subproblems, arbitrary depth of expressions
We show here how we can use this data structure and operation
framework for a very different purpose than what is usually 
the major use case for decision diagram technology, state-space exploration. 
We use the structure and natural caching mechanism to reduce the complexity of applying rewriting rules to an algebraic specification \cite{je-sais-pas}.
\fixme{YTM : heu je suis pas � l'aise l� dessus, un peu de texte de Didier ?}

% plan

Section \ref{sec:def} recalls the essential definitions. 
Section \ref{sec:cod} describes  the SDD encoding chosen and validates it's canonical form. 
Section \ref{sec:hom} describes the encoding of the rewriting rules as homomorphisms operating over the SDD. 
Section \ref{sec:perf} reports on some experimental results obtained with our prototype tool, \emph{Terms-SDD}. \fixme{trouver un nom}. 
Finally we conclude in section \ref{sec:conclu}.



