\section{Definitions}
\label{sec:def}

\subsection{SDD definition}

BDD are a powerful data structure, but their expressive power is
limited, and they are not very good at representing dynamic behavior.
In Set Decision Diagrams \cite{forte05}, an arc of the structure is
labeled by a
\emph{set} of values, instead of a single valuation. The set may itself
represented by an SDD, thus in effect we label the arcs of our
structure with references to SDD, introducing hierarchy in the data
structure. At the final depth level we use plain arrays of integers to
label the edges.


%definition des SDD

Set Decision Diagrams (SDD) are data structures for representing
sequences of assignments of the form $e_1 \in a_1 ;e_2 \in a_2;
\cdots e_n \in a_n$ where $e_i$ are variables and $a_i$ are sets of values.
We assume no variable ordering, and the same variable can occur
several times in an assignment sequence.  We also make no assumptions
on the domain of the variables. We encode SDD as shared decision
trees.  We define the usual terminals $0$ and $1$ to represent
non-accepting and accepting sequences respectively.  In the following,
$E$ denotes a set of variables, and for any $e$ in $E$, $\Dom(e)$
represents the domain of $e$,

\begin{definition}[Set Decision Diagram]\label{SDD_Def}
The set $\SDD$ of SDD is defined by $d\in \SDD$ if:
\begin{itemize}
\item $d \in \{0,1\}$ or
\item $d=(e,\alpha)$ with:
 \begin{itemize} \item $e \in E$ \item $\alpha$: is a finite set of
 pairs $(a_i,d_i)$ where $a_i \subseteq \Dom(e)$ and $d_i \in \SDD$
\end{itemize}
\end{itemize}
We denote $e \fireseq{a_i} d_i$, the SDD $(e,\alpha)$ with
$\alpha(a_i)=d$ and for all $a_j \neq a_i$, $\alpha(a_j)=0$. We call
SDD sequence an SDD of the form $e_1 \fireseq{a_1} e_2 \fireseq{a_2}
\ldots 1$ where $\forall i, |a_i|=1$.
\end{definition}

We further introduce a canonical representation for SDD, essential to
allow use of a unicity table and cache.  The SDD we manipulate are
canonized by construction, through the union operation given in
proposition \ref{prop:SDD_mapping} below.

\begin{definition}[Canonical Set Decision Diagram]\label{SDD_Can_Def}
An SDD $d$ is said to be canonical if and only if :
\begin{itemize} 
	\item $d = 0$ or $d=1$ \item $d = (e,\alpha)$ and $\forall
	(a_i,d_i),(a_j,d_j) \in \alpha, i \neq j$, $\left\{
\begin{tabular}{rl}
 $ 1.$ & $ a_i \cap a_j = \emptyset$ \\ $ 2.$ & $d_i \neq d_j$ \\ $
 3.$ & $a_i \neq \emptyset$ and $d_i \neq 0$
\end{tabular}
\right.$
\end{itemize}
\end{definition}

Intuitively this definition sets the constraints that:
\begin{enumerate}
\item The number of sets of values that are mapped to a non-zero SDD
be finite. This is required so that the number of arcs leading from a
node be finite, since \emph{only the arcs labeled with sets that map
to a non-zero SDD are stored in the data structure};
\item For a value $x$ of $\Dom(e)$, at most one non zero SDD is
associated. In other words the sets referenced on the arcs outgoing
from a node are disjoint. This is required to allow existence of a
unique canonical representation of sets, hence unicity and comparison
of SDD nodes.
\item No two arcs from a node may lead to the same SDD. This is the
crucial point, any time we are about to construct $e
\fireseq{a} d+ e \fireseq{a'} d$, we will construct
$e \fireseq{a \cup a'} d$ instead. This corresponds to fusing arcs
that would have led to the same node.
\item By definition, the empty set maps to 0 and is not represented.
\end{enumerate}

Some immediate effects of this definition should be highlighted :
\begin{itemize}
\item This definition assumes that sets of values can be (efficiently)
represented, as an arc of the shared decision tree representing the
SDD is labeled with a set of values. As an SDD itself represents a
set, we can use variables of domain $\SDD$ itself, introducing
hierarchy in the data structure.
\item In practice, the requirements on the data sets that label the
arcs of an SDD are that they offer the usual set theoretic operations
(union, intersection and set difference) and the ability to compute a
hash key for the set stored. These requirements are captured by an
abstract interface class, thus labeling an SDD with any type of set
implementation is easy.  We use a shared array implementation, with
copy on write semantics, backed by a unicity table.
\end{itemize}     

To handle paths of variable lengths, SDD are required to represent a
set of compatible assignment sequences. An operation over SDD is said
partially defined if it may produce incompatible sequences in the
result.

\begin{definition}[Compatible SDD sequences]
\label{SDD:def-well-defined} 
\begin{itemize}
\item Any SDD sequence is compatible with itself.
\item $1$ and $e \fireseq{a} d$ are incompatible
\item $e\fireseq{a}d$ and $e'\fireseq{a'}d'$ are compatible if 
$\left\{\begin{array}{l} e=e' \\
\land a $ and $a'$ are compatible$ \\
\land (a=a' \Rightarrow d $ and $ d' $are compatible$) \\ 
\end{array}
\right.$\\
\end{itemize}
The compatibility of $a$ and $a'$ is defined as SDD compatibility if
$a,a' \in \SDD$. If $a,a' \subset \Nat$ they are compatible, and a
subset of $\Nat$ is incompatible with a SDD. In essence compatibility of
$a$ and $a'$ means $a \setminus a', a \cup a'$, and $a \cap a'$ are
well-defined.
\end{definition}


\subsection{Operations on SDD}

\subsubsection{Set theoretic operations}

First, we generalize the usual set-theoretic operations -- sum
(union), product (intersection) and difference -- to sets of set
assignment sequences expressed in terms of SDD.

\begin{definition}[Compatible SDD set theoretic operations]
By definition, set theoretic operations are only offered over
compatible SDD.
\begin{itemize}
\item $0 + d = d + 0 = d$,$0 * d = d * 0 = 0$, $0 \setminus d = 0$ and $d \setminus 0 = d, \forall d \in \SDD$;
\item $1 + 1 = 1 * 1 = 1$, $1 \setminus 1 = 0$
\item $\tuple{e,\alpha} \diamond \tuple{e,\alpha'} = \tuple{e,\alpha \diamond \alpha'}, \forall \diamond \in \{+,*,\setminus\}$ 
\end{itemize}
\end{definition}

\subsubsection{SDD homomorphisms}

By analogy with DDD, SDD allow the definition of user defined
operations through a recursive and compositional definition :
inductive homomorphisms. The essential constraint over homomorphisms
is linearity over the set of sequences contained in an
SDD. Homomorphisms can then be combined by sum and composition.

 \begin{definition}[Homomorphism] A mapping $\Phi$ on SDD is a fully
 defined \emph{homomorphism} if $\Phi(0)=0$ and $\forall d_1,d_2 \in
 \SDD: \Phi(d_1)+\Phi(d_2) = \Phi(d_1+d_2) $
\end{definition}


\begin{proposition}[Sum and composition]
\label{SDD:prop-hom-operator} Let $\Phi_1$, $\Phi_2$ be two
homomorphisms. Then $\Phi_1 + \Phi_2$, $\Phi_1 \circ \Phi_2$ are
homomorphisms.
\end{proposition}

The \textbf{transitive closure} $^*$ is also introduced, and allows to
perform a local fixpoint computation. It is defined thus~: for a
homomorphism $h$, $h^*(d)$ is computed by repeating $ d \leftarrow
h(d)$ until a fixpoint is reached. We often use $(h+ \mathit{Id})^*$
in our fixpoint computations. From here we can allow the definition of
user-defined inductive homomorphisms:

\begin{proposition}[Inductive homomorphism]
\label{SDD:prop-hom-equation} 
The following recursive definition of mappings $(\Phi_k)_{k}$ defines
a family of homomorphisms called inductive homomorphisms.:
\[\begin{array}{lcl}
\forall d\in \SDD, \Phi_k(d)&=& \left\{\begin{array}{ll}
 0 & \mbox{ if } d = 0 \\ d' \in \SDD & \mbox{ if } d = 1 \\
% \top & \mbox{ if } d = \top \\ 
\alpha' = \sum_{i=1}^k  \Phi_{k}(e,a_i) (d_i)
& \mbox{ if } d = (e,\alpha = \bigcup_{i=1}^k \{ a_i\rightarrow d_i\})
\\
\end{array}
\right. \\
\end{array}
\]
$\Phi_{k} (e,a)$ is inductively defined as a sum $ \Phi_{k} (e,a) =
\sum_{l} \pi_{l}(e,a) \circ \Phi_{l} + \pi_{0}(e,a) $ where all
$\pi_{l}(e,a)$ are SDD homomorphisms, linear over the elements of $a$
($\forall a,a' \subseteq \Dom(e):\pi_l(e,a \cup a') = \pi_l(e,a) +
\pi_l(e,a')$).
\end{proposition}


To define a family of inductive homomorphisms $\Phi$, one has just to
set the homomorphisms for the symbolic expression $\Phi (e,a_i)$ for
any variable $e$ and set $a_i$ and the SDD $\Phi(1)$. It should be
noted that this definition differs from the DDD inductive homomorphism
in that $\Phi(e,a_i)$ is defined over the
\emph{sets} ($a_i \subseteq \Dom(e)$) of values of the variable $e$'s domain
$\Dom(e)$. This is a fundamental difference as it requires $\Phi$ to
be defined in an ensemblist way: we cannot by this definition define
the evaluation of $\Phi$ over a single value of $e$. However $\Phi$
must be defined for the \emph{set} containing any single value.

In addition $\pi(e,a)$ must be an SDD homorphism linear over the
element of $a$.

We use most commonly homomorphisms of the form $e \fireseq{\phi(a)}
Id$ which allows a linear operation on the values labeling the arc,
and by composition with another inductive homomorphism, to realize an
operation on the rest of the paths of the SDD.

