\section{Encoding the transition relation}
\label{sec:hom}


\subsection{Na�ve approach (v1)}

A first analysis of the problem can suggest an encoding of the
transition relation in a generic way by using a homomorphism taking
three parameters : the disk or ring to be moved $r$, the origin pole $ori$ and
the destination pole $dest$. This allows to encode all possible moves
 quite easily, although the number of moves is $d \cdot p \cdot (p-1)$
 ($d$ disks, $p \cdot (p-1)$ potential moves per disk). This setting remains
 reasonable provided $p$ does not grow too much ; in the classical problem
 with $p=3$ we have $6$ moves per disk.

We obtain the following form for this homomorphism :

$$
\begin{array}{lcl}
swap\_pole\tuple{r,ori,dest} (e,x)= \\
\left\{\begin{array}{ll}
 e \fireseq{x} \mathit{this} & \mbox{ if } e \neq r \\
 e \fireseq{dest} \mathit{noring}\tuple{ori,dest} & \mbox{ if } e = r \land x = ori \\
 0  & \mbox{ otherwise } \\
\end{array}
\right. \\
swap\_pole\tuple{r,ori,dest} (1) = 1 
\end{array}
$$

The tester homomorphism $noring$ is used to ensure that that the movement is legal, 
 i.e. that all rings below (smaller) than the target ring are neither on $ori$ nor on $dest$.
It is written thus :

$$
\begin{array}{lcl}
 \mathit{noring}\tuple{ori,dest} (e,x)= \\
\left\{\begin{array}{ll}
 e \fireseq{x} \mathit{this} & \mbox{ if } x \neq ori \land x \neq dest \\
 0  & \mbox{ otherwise } \\
\end{array}
\right. \\
 \mathit{noring}\tuple{ori,dest} (1) = 1 
\end{array}
$$

Since the behavior is symmetric, we can additionally request that $ori < dest$, and enforce this constraint in the homomorphisms constructor.
This reduces the number of instances of this homomorphism that nees to be created
 to $\Sigma_{i=1}^p p-i=\Sigma_{i=0}^{p-1} i= \frac {p \cdot (p-1)}{2}$.

We construct and store the parametered $swap\_pole\tuple{r,ori,dest}$ homomorphisms in a vector called $events$. 

Finally we perform a BFS style fixpoint, with state accumulation.
\begin{lstlisting}
  // Fixpoint over events + Id
  DDD ss, tmp = M0;
  do {
    ss = tmp;
    vector<Hom>::iterator it = events.begin();
    for ( ; it != events.end(); ++it) {
      tmp = tmp + (*it) (tmp);
    }
  } while (ss != tmp);
\end{lstlisting}

The results obtained (see performances in section \ref{sec:perfs})
show an exponential blowup of the number of intermediate nodes and of
the cache usage. Increasing the number of poles makes things much
worse, due to the $p \cdot (p-1)$ quadratic blowup in the number of
homomorphism instances, that is further reflected on the number of cache entries.


\subsection{Reducing the number of events (v2)}

A first improvement we can think of, is to try to reduce the number of
events.  In the previous algorithm, each event instance has to
traverse all the nodes from root to appropriate application level, and
then it might happen that the event is not enabled (e.g. the target
ring is not on the $ori$ pole).

This is very wasteful, time-wise (lots of useless traversals), and
memory-wise because homomorphism applications are cached, so the cache
will get very large.  We can write a better version of the
$move\_ring$ homomorphism, which decides which $dest$ we will work
with, only once the ring is reached (and $ori$ is read from the
current state).

It is written thus, where $noring$ is left unchanged from the previous version :
$$
\begin{array}{lcl}
move\_ring\tuple{r} (e,x)= \\
\left\{\begin{array}{ll}
 e \fireseq{x} \mathit{this} & \mbox{ if } e \neq r \\
\Sigma_{dest=1,dest\neq x}^{p} e \fireseq{dest} \mathit{noring}\tuple{x,dest} &  \mbox{ otherwise } \\
\end{array}
\right. \\
move\_ring\tuple{r} (1) = 1 
\end{array}
$$

The sum above is written in c++ as :
\begin{lstlisting}
GHom phi(int vr, int vl) const {
	/// example shortened    
      // ring reached 
      // try to move to all new positions
      GHom res = GDDD::null;
      for (int i=0 ; i <NB_POLES ; i++) {
	// test all possible moves from current position = vl
	if (i != vl) {
	  // update ring position and test no ring above
	  // no_ring_above propagates on the bottom of the DDD ;
	  // it returns 0 if preconditions are not met 
	  // or a DDD with only paths where the move is legal
	  res = res + ( GHom (ring_ , i) & new _no_ring_above(i , vl) );
	}
      }      
      return res ;
}
\end{lstlisting}

The impact is immediate, with much less intermediate nodes and cache
entries.  We measure a reduction factor with respect to $v1$ that
increases with $d$ the number of disks, with a factor of roughly $2$
when $d=5$, and a factor of $5$ for $d=10$.  However peak size still
progresses exponentially with the number of rings.

\subsection{Moving the union down  (v3)}

We can further improve our algorithm if we take into account that what
we are interested in the full reachable state-space. 
The issue is that in the BFS loop, we first fire an event, then union
the result with the previously reached states. 
This treatment is done externally, from the main loop, thus the union
has to travel from the root back to where the update has taken place.
This also creates useless intermediate nodes (just before union) that
we can be pretty sure will not appear in the result.

We can slightly rewrite the $move\_ring$ homomorphism to keep
previously existing paths and only add to them. Note however that
applying our event now will NOT produce only the successor states,
 which does not make it fit for more complex problems than
 reachability such as CTL or counter-example search.

$$
\begin{array}{lcl}
move\_ring\tuple{r} (e,x)= \\
\left\{\begin{array}{ll}
 e \fireseq{x} \mathit{this} & \mbox{ if } e \neq r \\
 e \fireseq{x} \mathit{Id} + \Sigma_{dest=1,dest\neq x}^{p} e \fireseq{dest} \mathit{noring}\tuple{x,dest} &  \mbox{ otherwise } \\
\end{array}
\right. \\
move\_ring\tuple{r} (1) = 1 
\end{array}
$$

We can rewrite the BFS loop to remove the union with previous ($tmp = tmp + (*it) tmp \leadsto tmp = (*it) tmp$ states.

This optimization reduces roughly by a factor of $2$ the cache and peak size, with respect to $v2$.
However, we still grow exponentiallly with the number of disks. 


\subsection{Introducing saturation (v4)}

In this version, we will introduce the recursive firing algorithm due
to Ciardo et al. \cite{CiardoTacas03,ciardo01saturation}. The
principle is to fire events bottom up, by executing nested fixpoint
iterations.

$$
\begin{array}{lcl}
move\_ring\tuple{r} (e,x)= \\
\left\{\begin{array}{ll}
 e \fireseq{x} \mathit{this} & \mbox{ if } e \neq r \\
( e \fireseq{x} \mathit{Id} + \Sigma_{dest=1,dest\neq x}^{p} e \fireseq{dest} \mathit{noring}\tuple{x,dest}) \circ \\
 move\_ring\tuple{r-1}^* &  \mbox{ otherwise } \\
\end{array}
\right. \\
move\_ring\tuple{r} (1) = 1 
\end{array}
$$


The addition of the fixpoint application $move\_ring\tuple{r-1}^*$
allows to ensure that the successor node has already explored all its
possible states before continuing the evaluation.

The gain is awesome, resulting memory and time complexity becomes
linear to the number of disks, with $22$ new nodes and $39$ new cache
entries when $d$ is increased by $1$.

