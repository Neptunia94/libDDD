\section{Encoding the states using SDD}
\label{sec:cod}

Let $p$ designate the number of poles or towers, and $d$ designate the
number of disks.

We essentially have two possible encodings of the state :
\begin{itemize}
\item We use $p$ variables $v_1 .. v_p$ representing the poles, the domain of each
variable will be states a pole can reach, which is bounded by $2^d$
(each disk is on the pole or not). It is immediately obvious that this
high upper bound may lead to a huge number of arcs. Furthermore, we
have a strong dependency between our variables in this schema, since a
given disk can only be on one pole at a given time. These are bad
characteristics of an encoding.
\item We use $d$ variables $v_1..v_d$ representing the disks, each one has a domain
$1..p$ which tells which pole the disk is on in this state. This
encoding puts reasonable bounds on the number of arcs per node, and
promotes the relative independence of the different disk states. This
is the encoding we will use.
\end{itemize}

The second problem is variable order, which we know is often critical
when using decision diagrams. There does not seem to be reason to
shuffle the variables, as the problem is very symmetric. If we chose a
natural order for disks, we still have the choice of putting the
smallest disk at the top or at the bottom of the DD. 
Although it may seem there is little difference between the two, much
depends on the algorithm we will use to generate the state-space.

For BFS style iterations it does not really matter which disk is at
the top of structure, however, saturation can profit from having the
smaller disk at the bottom of the structure. Indeed, a movement of a
given disk is only constrained by the state of the smaller disks. Thus
the smallest disk has no constraints on its movement in any
state. Putting smaller disks near the leaves of the structure will
help discover states faster as we will see.

The initial state is thus encoded by the DDD : \\
$v_d \fireseq{0} \ldots v_2 \fireseq{0} v_1 \fireseq{0} 1$ \\
where $v_d$ represents the state of the largest disk. 
